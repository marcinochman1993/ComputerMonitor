\documentclass[a4paper]{article}

\usepackage[utf8]{inputenc}
\usepackage[polish]{babel}
\usepackage{polski}
\usepackage[margin=0.9in]{geometry}
\usepackage{tabularx}

\author{Marcin Ochman \\* \textit{200546}}
\title{ Projekt z Wizualizacji Danych Sensorycznych \\* 
	\huge{\uppercase{\textbf{Computer Monitor}}}}
\date{}

\begin{document}
	\maketitle
	
	\section{Osoby uczestniczące w projekcie}
		Projekt zostanie wykonany przez \textit{Marcina Ochmana}.
	
	
	\section{Założenia projektowe}
		Celem projektu jest napisanie aplikacji \textit{,,Computer Monitor''} przeznaczoną na komputery osobiste, która:
	\begin{itemize}
		\item pozwala na monitorowanie pracy komputera(ów), z którym(i) komputer jest połączony poprzez sieć
		\item pracuje w różnych trybach:
		\begin{itemize}
			\item tryb wizualizacji
			\item tryb raportowania
			\item połączenie dwóch poprzednich trybów
		\end{itemize}
		\item potrafi zapamiętać ustawienia użytkownika, w tym m.in. tryb działania aplikacji, układy okien, ustawienia wykresów itp.
		\item posiada przejrzysty, intuicyjny, atrakcyjny oraz konfigurowalny interfejs użytkownika rozumiany w następujący sposób:
		\begin{itemize}
			\item będą wyraźnie zaznaczone różne kategorie monitorowanych wielkości np. procesor, karta graficzna, pamięć \uppercase{ram}
			\item zaimplementowanie animacji
			\item dostosowywanie wykresów np. skala osi czasu
			 
		\end{itemize}
		\item pomimo realizowanej komunikacji poprzez sieć oraz innych zadań, jest wysoce 
		interaktywna
		\item docelową platformą jest system Linux, jednak zostaną poczynione wszelkie starania, aby aplikacja działała również na systemach Windows
	\end{itemize}
	
	\section{Opis poszczególnych trybów pracy}
	Poniżej zostały opisane poszczególne tryby pracy aplikacji.
	
	\subsection{Tryb wizualizacji}
		W tym trybie program prezentuje dane, które są pobierane poprzez sieć. Pozwala na śledzenie wartości poszczególnych wielkości opisujących stan komputera oraz prezentowanie danych na wykresach (zależności czasowe, zużycie zasobów komputera np. w formie wykresu kołowego) oraz diagramach/ilustracjach (monitorowanie i alarmowanie poprawności funkcjonowania systemu). Progi alarmowe można również zdefiniować samodzielnie. Dodatkowo będzie możliwość zapisywania raportów zawierających dane zdobyte w trakcie monitorowania.
	
	\subsection{Tryb raportowania}
		Program działający w trybie raportowania ma za zadanie śledzenie pracy komputera oraz wysyłanie zebranych informacji poprzez sieć. Użytkownik ma możliwość wyboru danych, które zostaną wysłane. Aplikacja będzie działać w systemowym zasobniku.
	
	\subsection{Tryb łączony}
		Jest to połączenie dwóch poprzednich trybów. Pozwala na jednoczesne wizualizowanie aktualnego stanu komputera, na którym została uruchomiona aplikacja oraz wysyłanie zebranych informacji poprzez sieć. Dwa ostatnie tryby zostały wyodrębnione ze względu na umożliwienie optymalizacji zasobów zajmowanych przez aplikację.
	
	\section{Kamienie milowe}
		Do ukończenia aplikacji należy zaliczyć kolejne kamienie milowe, bez których aplikacja nie mogłaby funkcjonować jako całość. Przedstawiają się one następująco
		
		\begin{table}[h]
			\centering
			\begin{tabularx}{0.65\textwidth}{|c|X|}
				\hline
				Lp. & Opis \\ \hline
				1 & Stworzenie biblioteki, która pozwoli na pobieranie informacji o stanie komputera \\ \hline
				2. & Zaprogramowanie biblioteki służącej do komunikacji komputerów w siecie w celu wysyłania informacji zgromadzonych przez bibliotekę z pkt 1 \\ \hline
				3. & Zaprojektowanie oraz oprogramowanie interfejsu użytkownika \\ \hline
				4. & Połączenie pracy bibliotek z interfejsem użytkownika \\ \hline
			\end{tabularx}
			\caption{Tabela przedstawiająca kamienie milowe}
		\end{table}
		
			\section{Efekty}
			Dzięki pracy nad projektem można uzyskać:
			\begin{itemize}
				\item praktyczną wiedzę na temat programowania aplikacji graficznych
				\item doświadczenie w programowaniu sieciowym
				\item umiejętność współpracy z systemem operacyjnym oraz wykorzystywania jego API 
			\end{itemize}
		
		\section{Harmonogram}
		
		W tabeli \ref{tabela_harmonogram} przedstawiono plan pracy nad aplikacją z dokładnością do jednego tygodnia.
		
		\begin{table}[h]
			\centering
			\begin{tabularx}{0.8\textwidth}{|c|c|c|X|}
				\hline
				Od & Do & Osiągnięty kamień milowy & Opis \\ \hline
				23.03 & 29.03 &  & Zaprojektowanie interfejsu biblioteki do monitorowania komputera \\ \hline
				30.03 & 05.04 &  & Zapoznanie się z metodami dostępu do informacji o komputerze oraz rozpoczęcie implementacji powyższej biblioteki\\ \hline
				06.04 & 12.04 & 1 & Ukończenie biblioteki monitorującej komputer  \\ \hline
				13.04 & 19.04 &   & Rozpoczęcie projektowania interfejsu biblioteki do komunikacji sieciowej \\ \hline
				20.04 & 26.04 &  & Zapoznanie się z metodami programowania sieciowego i dostępnymi bibliotekami oraz  rozpoczęcie implementacji biblioteki sieciowej\\ \hline
				27.04 & 03.05 &  & Implementacja biblioteki sieciowej \\ \hline
				04.05 & 10.05 & 2 & Zakończenie prac nad biblioteką sieciową oraz rozpoczęcie projektowania interfejsu użytkownika \\ \hline
				11.05 & 17.05 &    & Kontynuacja projektowania interfejsu użytkownika oraz jego implementacja \\ \hline
				18.05 & 24.05 &  & Implementacja interfejsu użytkownika \\ \hline
				25.03 & 31.03 & 3 & Implementacja interfejsu użytkownika oraz połączenie logiki programu z interfejsem \\ \hline
				01.06 & 07.06 & 4 & Łączenie aplikacji graficznej z bibliotekami oraz początek testów aplikacji \\ \hline
				07.06 & 13.06 &  & Ostateczne testy aplikacji \\ \hline
			\end{tabularx}
			\caption{Tabela przedstawiająca kamienie milowe}
			\label{tabela_harmonogram}
		\end{table}
		
\end{document}