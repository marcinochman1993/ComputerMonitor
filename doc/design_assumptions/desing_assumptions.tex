\documentclass[a4paper]{article}

\usepackage[utf8]{inputenc}
\usepackage[polish]{babel}
\usepackage{polski}
\usepackage[margin=0.9in]{geometry}

\author{Marcin Ochman \\* \textit{200546}}
\title{ Projekt z Wizualizacji Danych Sensorycznych \\* 
	\huge{\uppercase{\textbf{Computer Monitor}}}}
\date{}

\begin{document}
	\maketitle
	
	\section{Osoby uczestniczące w projekcie}
		Projekt zostanie wykonany przez \textit{Marcina Ochmana}.
		
	\section{Założenia projektowe}
	Celem projektu jest napisanie aplikacji \textit{,,Computer Monitor''} przeznaczoną na komputery osobiste, która:
	\begin{itemize}
		\item pozwala na monitorowanie pracy komputera(ów), z którym(i) komputer jest połączony poprzez sieć
		\item pracuje w różnych trybach:
		\begin{enumerate}
			\item tryb wizualizacji
			\item tryb raportowania
			\item połączenie dwóch poprzednich trybów
		\end{enumerate}
		\item potrafi zapamiętać ustawienia użytkownika, w tym m.in. tryb działania aplikacji
		\item posiada przejrzysty, intuicyjny, atrakcyjny oraz konfigurowalny interfejs użytkownika
		\item pomimo realizowanej komunikacji poprzez sieć, jest wysoce 
		interaktywna
		\item docelową platformą jest system Linux, jednak zostaną poczynione wszelkie starania, aby aplikacja działała również na systemach Windows
	\end{itemize}
	
	\section{Opis poszczególnych trybów pracy}
	Poniżej zostały opisane poszczególne tryby pracy aplikacji.
	
	\subsection{Tryb wizualizacji}
		W tym trybie program prezentuje dane, które są pobierane poprzez sieć. Pozwala na śledzenie wartości poszczególnych wielkości opisujących stan komputera oraz prezentowanie danych na wykresach - zależności czasowe oraz diagramach - monitorowanie i alarmowanie poprawności funkcjonowania systemu, zużycie zasobów komputera. Progi alarmowe można również zdefiniować samodzielnie.
	
	\subsection{Tryb raportowania}
		Program działający w trybie raportowania ma za zadanie śledzenie pracy komputera oraz wysyłanie zebranych informacji poprzez sieć. Użytkownik ma możliwość wyboru danych, które zostaną wysłane. Aplikacja będzie działać w systemowym zasobniku.
	
	\subsection{Tryb łączony}
		Jest to połączenie dwóch poprzednich trybów. Pozwala na jednoczesne wizualizowanie aktualnego stanu komputera, na którym została uruchomiona aplikacja oraz wysyłanie zebranych informacji poprzez sieć.
	
	
\end{document}