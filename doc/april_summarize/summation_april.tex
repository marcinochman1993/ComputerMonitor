\documentclass[a4paper]{article}

\usepackage[utf8]{inputenc}
\usepackage{polski}
\usepackage[polish]{babel}
\usepackage[margin=1in]{geometry}
\usepackage{lmodern}
\usepackage{tabularx}
\usepackage{graphicx}
\newcounter{counter}
\newcommand\rownumber{\stepcounter{counter}\arabic{counter}}

\title{\textbf{Podsumowanie prac z projektu WDS - kwiecień 2015}}
\author{Marcin Ochman - 200546}
\date{}
\begin{document}


\begin{titlepage}
\begin{center}

 \newcommand{\HRule}{\rule{\linewidth}{0.5mm}}
%\includegraphics[width=0.15\textwidth]{./logo}~\\[1cm]

\textsc{\Large Projekt z Wizualizacji danych sensorycznych}\\[1cm]

\includegraphics{network_icon}


\HRule \\[0.4cm]
{ \huge \bfseries \textit{Computer Monitor} - monitorowanie komputerów poprzez sieć.\\[0.4cm] }

{\huge \underline{\textit{Raport postępu prac }}}\\[0.5cm]


\LARGE 14.05.2015
\HRule \\[1.5cm]


\noindent
\begin{minipage}[t]{0.4\textwidth}
\begin{flushleft} \large
\emph{Autor:}\\
Marcin \textsc{Ochman}
\end{flushleft}
\end{minipage}%
\begin{minipage}[t]{0.4\textwidth}
\begin{flushright} \large
\emph{Prowadzący} \\
Dr inż. Bogdan \textsc{Kreczmer}
\end{flushright}
\end{minipage}


\end{center}
\end{titlepage}

\newpage

\tableofcontents
\listoffigures
\listoftables

\newpage

\section{Wprowadzenie}

W ciągu miesiąca tj. 18.03-23.04 zostało wykonanych wiele prac nad aplikacją. W tym dokumencie opisano wykonane i przewidywane na następny miesiąc zadania oraz komentarz autora na temat postępów prac nad aplikacją.

\section{Lista wykonanych zadań w projekcie}

W poniższej tabeli zostały zebrane zadania, które udało się zrealizować do dnia 23.04.2015r.

\begin{table}[h]
\centering
\begin{tabularx}{0.7\linewidth}{ |c|X| }
			\hline 
			\rownumber & Opracowano architekturę aplikacji - użyte biblioteki oraz 
						 klasy do napisania\\ \hline
			\rownumber & Utworzono strukturę katalogów aplikacji \\ \hline
			\rownumber & Aplikacja buduje się przy pomocy wieloplatformowego narzędzia 
						 \textit{CMake} - napisanie plików potrzebnych do poprawnej kompilacji \\ \hline
			\rownumber & Rozpoczęto pracę nad biblioteką \textit{SystemMonitoringLib} \\ \hline
			\rownumber & Rozpoczęto pracę nad interfejsem użytkownika \\ \hline
			\rownumber & Rozpoczęta pracę nad komunikacją pomiędzy biblioteką \textit{SystemMonitoringLib} oraz interfejsem użytkownika \\ \hline
	\end{tabularx}
	\caption{Tabela prac nad aplikacją}
\end{table}

\section{Szczegółowy opis wykonanych zadań}

\subsection{Opracowanie architektury aplikacji}
Program będzie opierać się na bibliotekach \textit{Qt5} i \textit{QCustomPlot}, które posłużą do prezentacji danych oraz dwóch autorskich bibliotek, które zostały opisane w następnych punktach: \textit{SystemMonitoringLib}. Biblioteki monitorujące będą komunikować się z warstwą prezentacji za pomocą odpowiedniej klasy pośredniczącej. Aplikacja została napisana przy wykorzystaniu wzorca projektowego \textit{MVC}. Diagram budowy aplikacji został przedstawiony na rysunku \ref{diagram_budowy_aplikacji}, a diagramy klas zostały przedstawione na rysunkach \ref{diagram_klas_system_monitoring} oraz \ref{diagram_klas_gui}.

\begin{figure}[h]
	\centering
	\includegraphics[width=\linewidth]{img/diagramBudowyAplikacji.png}
\end{figure}


\begin{figure}[h]
	\centering
	\includegraphics[width=0.75\paperheight, angle=90]{img/diagramKlas.png}
	\caption{Diagram UML stworzonych klas niezwiązanych z interfejsem użytkownika}
	\label{diagram_klas_system_monitoring}
\end{figure}

\begin{figure}[h]
	\centering
	\includegraphics[width=0.75\paperheight, angle=90]{img/diagramKlasGui.png}
	\caption{Diagram UML stworzonych klas}
	\label{diagram_klas_gui}
\end{figure}

\subsection{Stworzenie zalążka aplikacji}

Zalążek aplikacji wymagał kilku pomniejszych kroków do wykonania:
\begin{itemize}
	\item utworzenie struktury folderów projektu
	\item napisanie skryptów kompilujących i konsolidujących aplikację
\end{itemize}
Opis poszczególnych zadań znajduje się poniżej.

\subsubsection{Struktura folderów projektu}
W głównym folderze aplikacji znajdują się dwa foldery: \textit{doc} oraz \textit{prj}. Pierwszy zawiera wszelkie pliki przechowujące informacje o dokumentacji, a drugi zawiera kody źródłowe aplikacji. W folderze \textit{prj} można wyróżnić kilka folderów. Ich krótki opis zebrano w tabeli \ref{opis_folderow_pr}.

\begin{table}
\centering
\begin{tabularx}{0.7\linewidth}{|c|X|}
	\hline
	inc & przechowuje wszystkie pliki nagłówkowe, które nie należą do bibliotek \\ \hline
	src & zawiera pliki źródłowe, które nie należą do bibliotek \\ \hline
	lib & przechowuje pliki źródłowe oraz nagłówkowe bibliotek, każda biblioteka jest umieszczona 
		  w innym folderze. W każdym z folderów jest są pliki nagłówkowe oraz źródłowe \\ \hline
	ui & znajdują się w nim wszystkie pliki  programu QtDesigner \\ \hline
	rsrc  & zawiera zasoby aplikacji \\ \hline
\end{tabularx}

\label{opis_folderow_prj}
\caption{Opis poszczególnych folderów w katalogu \textit{prj/}}
\end{table}

\subsubsection{Proces kompilacji oraz opis narzędzia \textit{CMake}}
Program \textit{CMake} to wieloplatformowy system do budowania aplikacji. Dzięki niemu łatwe staje się
znajdowanie odpowiednich plików nagłówkowych oraz bibliotek oraz kompilacja na różnych platformach (m.in. Linux i Windows). Każdy folder zawiera w sobie plik \textit{CMakeLists.txt}, który definiuje w jaki sposób ma zachować się program budujący.Dzięki niemu, oszczędzono wiele czasu na kompilację oraz konsolidację programu używającego bibliotek \textit{Boost} oraz \textit{Qt5}. Najpierw budowane są biblioteki, a następnie cały program. Wszystko zostaje skonsolidowane. Istnieje również możliwość wygenerowania dokumentacji, wystarczy, że zbudujemy cel \textit{doc}.

\subsection{Biblioteka \textit{SystemMonitoringLib}}
Jest to biblioteka do monitorowania lokalnego komputera oraz wysyłania informacji przez sieć. Pozwala na pobieranie odczytów z czujników, informacji o procesorze takich jak częstotliwość taktowania poszczególnych rdzeni, zużycie procesora, o pamięci RAM (dostępna pamięć, zużycie) oraz informacje o poszczególnych procesach. Na komputerze z uruchominionym systemem \textit{Linux}, biblioteka do zbierania informacji o komputerze używa innej biblioteki - \textit{sensorslib} oraz przetwarze pliki m.in. \textit{/proc/stat/}, \textit{/proc/cpuinfo}

\subsection{Interfejs użytkownika}
Roczpoczęto prace nad wizualizowaniem danych  odczytanych z czujników dostępnych na płycie głównej.
W widoku tabelki mamy możliwość podglądania aktualnych wartości odczytanych wielkości. Zrzut ekranu przedstawiającego aktualny stan interfejsu użytkownika został pokazany na rysunku \ref{wygladAplikacji}.

\begin{figure}[h]
	\centering
	\includegraphics[width=\linewidth]{img/wygladAplikacji.png}
	\caption{Wygląd aplikacji \textit{ComputerMonitor}}
	\label{wygladAplikacji}
\end{figure}



\section{Następne zadania do wykonania - aktualizacja planu pracy}
Ze względu na to, że niektóre zadania zostały wykonane ponad plan,a pewna część zadań z planu została nie zrealizowana należało zastosować poprawki w planie pracy nad aplikacją. Tabela \ref{planPracy} przedstawia zaktualizowany plan pracy.

\begin{table}[h]
			\centering
			\begin{tabularx}{0.8\textwidth}{|c|c|c|X|}
				\hline
				Od & Do & Osiągnięty kamień milowy & Opis \\ \hline
				20.04 & 26.04 &  & Zapoznanie się z metodami programowania sieciowego i dostępnymi bibliotekami oraz  rozpoczęcie implementacji biblioteki sieciowej, prace nad biblioteką monitorującą\\ \hline
				27.04 & 03.05 & 1 & Implementacja biblioteki sieciowej oraz finalizacja prac nad biblioteką monitorującą \\ \hline
				04.05 & 10.05 & 2 & Zakończenie prac nad biblioteką sieciową oraz kontynuacja projektowania interfejsu użytkownika \\ \hline
				11.05 & 17.05 &    & Dalsze projektowanie interfejsu użytkownika oraz jego implementacja \\ \hline
				18.05 & 24.05 &  &  \\ \hline
				25.03 & 31.03 & 3 & Implementacja interfejsu użytkownika oraz połączenie logiki programu z interfejsem \\ \hline
				01.06 & 07.06 & 4 & Łączenie aplikacji graficznej z bibliotekami oraz początek testów aplikacji \\ \hline
				07.06 & 13.06 &  & Ostateczne testy aplikacji \\ \hline
			\end{tabularx}
			\caption{Tabela przedstawiająca kamienie milowe}
			\label{tabela_harmonogram}
		\end{table}

\section{Podsumowanie}
Ze względu na dużą liczbę innych projektów, prace nad aplikacją są lekko opóźnione - około tygodnia, jednak zostaną poczynione wszelkie starania aby do następnego terminu oddania prac wszystko zostało nadrobione. 
Pozytywną informacją jest fakt, że zostały już rozpoczęte prace nad interfejsem użytkownika, czego początkowy plan pracy nie uwzględniał.

\end{document}